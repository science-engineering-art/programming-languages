\documentclass[10pt]{amsart}
\setcounter{tocdepth}{3}% to get subsubsections in toc

\let\oldtocsection=\tocsection

\let\oldtocsubsection=\tocsubsection

\let\oldtocsubsubsection=\tocsubsubsection

\renewcommand{\tocsection}[2]{\hspace{0em}\oldtocsection{#1}{#2}}
\renewcommand{\tocsubsection}[2]{\hspace{1em}\oldtocsubsection{#1}{#2}}
\renewcommand{\tocsubsubsection}[2]{\hspace{2em}\oldtocsubsubsection{#1}{#2}}

\usepackage[utf8]{inputenc}

\usepackage[spanish]{babel}
\usepackage{blindtext}

\usepackage{listings}

\usepackage{amsmath}
\usepackage{amssymb}
\usepackage{amsfonts}
\usepackage{color}
\usepackage{hyperref}
\usepackage{url}
\usepackage{stmaryrd}
\usepackage{calrsfs}
\usepackage{fancyhdr}
\usepackage{textcomp}
\usepackage{graphicx}
\usepackage{stmaryrd}
\usepackage{lipsum}

\hypersetup{
	colorlinks=true,
	linkcolor=black,
	filecolor=magenta,      
	urlcolor=cyan,
	pdftitle={Overleaf Example},
	pdfpagemode=FullScreen,
}

% Define a custom color
\definecolor{backcolour}{rgb}{0.95,0.95,0.92}
\definecolor{codegreen}{rgb}{0,0.6,0}

% Define a custom style
\lstdefinestyle{myStyle}{
	backgroundcolor=\color{backcolour},   
	commentstyle=\color{codegreen},
	basicstyle=\ttfamily\footnotesize,
	breakatwhitespace=false,         
	breaklines=true,                 
	keepspaces=true,                 
	numbers=left,       
	numbersep=5pt,                  
	showspaces=false,                
	showstringspaces=false,
	showtabs=false,                  
	tabsize=2,
}

% Use \lstset to make myStyle the global default
\lstset{style=myStyle}

\voffset=-1.4mm
\oddsidemargin=14pt
\evensidemargin=14pt
\topmargin=26pt
\headheight=9pt     
\textheight=576pt
\textwidth=441pt 
\parskip=0pt plus 4pt

\title{Lenguajes de Programaci\'on}
\author{Seminario\#1\\
	 \textbf{C++11, C++14}.}


\begin{document}
	\begin{titlepage}
		\clearpage	
		\maketitle
		\begin{center}
			\includegraphics[width=10cm]{c++logo.png}
			
			\vspace{5em}
			Equipo 2:
			
			Marcos Manuel Tirador del Riego
			
			Laura Victoria Riera P\'erez
			
			Leandro Rodr\'iguez Llosa
			\vspace{1em}
			
			Grupo: C-311
		\end{center}
		\thispagestyle{empty}
	\end{titlepage}


\newpage
\pagenumbering{gobble}
\tableofcontents
\thispagestyle{empty}

\newpage
\pagenumbering{arabic}
\section{Definici\'on de las clases genéricas linked\_list y node}

\lstinputlisting[caption=Sample Code Listing C++, label={lst:listing-cpp}, language=C++]{sample1.cpp}

\newpage
\section{Definici\'on de miembros de datos necesarios de ambas clases}

\subsection{Nuevos elementos introducidos a partir de C++11 que permiten un manejo más "inteligente" de la memoria}

\subsection{Inicializaci\'on}

\subsection{Filosofía en el uso de la memoria defendida por C++}

\subsection{Simplificaci\'on de nombres de tipos mediante el uso de alias}

\newpage
\section{Definici\'on de  los constructores clásicos de C++(C++0x) , el constructor move y las sobrecargas del operador =}

\subsection{¿Qué hace cada uno de ellos? ¿Cuándo se llaman?}

\subsection{¿Qué es un lvalue y un rvalue ?}

\subsection{std::move}

\newpage
\section{Definici\'on de un constructor que permita hacer list-initialization lo más parecido a C\# posible}

\subsection{Compare la utilización del \{\} v.s ()}

\newpage
\section{Definici\'on de  un constructor que reciba un vector $ <T> $}

\subsection{Usar for\_each con expresiones lambda}

\newpage
\section{Definici\'on del destructor de la clase}

\subsection{¿Hace falta?}

\subsection{¿Para qué casos haría falta un puntero crudo (raw pointer)?}

\newpage
\section{Definici\'on de las funciones length , Add\_Last , Remove\_Last , At , Remove\_Ate}

\subsection{Noexcept}

\subsection{Inferencia de tipo en C++ (auto, decltype decltype(auto)). Explicar todos, pero no obligatoriamente usarlos.}

\newpage
\section{Crear un puntero a función Function<R, T...> que devuelve un valor de tipo R y recibe un número variable de parámetros de tipo T .}

\subsection{Definir una función genérica Map a linked\_list en T y R , que recibe un puntero a función que transforma un elemento T en uno R; de manera que Map devuelve una instancia de linked\_list$ < $R$ > $ resultado de aplicar a todos los elementos T de la lista original la función de transformación.}

\subsection{Crear punteros a funciones usando alias }

\subsection{Crear un puntero a función Function que permita cualquier cantidad de parámetros de cualquier tipo.}

\end{document}